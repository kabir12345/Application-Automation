Here is the updated LaTeX resume with the revised Technical Skills section:

```latex
\documentclass[10pt,letterpaper]{article}
\usepackage{latexsym}
\usepackage[empty]{fullpage}
\usepackage{titlesec}
\usepackage{marvosym}
\usepackage[usenames,dvipsnames]{color}
\usepackage{verbatim}
\usepackage{enumitem}
\usepackage[pdftex]{hyperref}
\usepackage{fancyhdr}
\usepackage[top=0.2in,bottom=0.2in,left=0.2in,right=0.2in]{geometry}
\usepackage{times}
\usepackage{enumitem}

\linespread{1}
\setlist[itemize]{itemsep=0pt}
\hypersetup{
  colorlinks=true,
  linkcolor=blue,
  filecolor=magenta,    
  urlcolor=blue,
  pdftitle={Kabir Jaiswal's Resume}
}

\pagestyle{fancy}
\fancyhf{}
\fancyfoot{}
\renewcommand{\headrulewidth}{0pt}
\renewcommand{\footrulewidth}{0pt}

\urlstyle{same}
\raggedbottom
\raggedright
\setlength{\tabcolsep}{0in}

\titleformat{\section}{
 \vspace{-13pt}\scshape\raggedright\large
}{}{0em}{}[\titlerule \color{black} \vspace{-4pt}]

\newcommand{\resumeItem}[2]{
  \item\small{
    \textbf{#1}{: #2 \vspace{-4pt}}
  }
}

\newcommand{\resumeSubheading}[4]{
  \vspace{-5pt}
    \begin{tabular*}{0.97\textwidth}{l@{\extracolsep{\fill}}r}
      \textbf{#1} & #2 \\
      \textit{\small#3} & \textit{\small #4} \\
    \end{tabular*}\vspace{-4pt}
}

\newcommand{\resumeProjectHeading}[2]{
  \vspace{-1pt}\item
    \begin{tabular*}{0.97\textwidth}{l@{\extracolsep{\fill}}r}
      \textbf{#1} #2 \\
    \end{tabular*}\vspace{-4pt}
}

\newcommand{\resumeSubItem}[2]{\resumeItem{#1}{#2}\vspace{-4pt}}

\renewcommand{\labelitemii}{$\circ$}

\newcommand{\resumeSubHeadingListStart}{\begin{itemize}[leftmargin=*]}
\newcommand{\resumeSubHeadingListEnd}{\end{itemize}}
\newcommand{\resumeItemListStart}{\begin{itemize}}
\newcommand{\resumeItemListEnd}{\end{itemize}}

\begin{document}

\begin{center}
  \textbf{\Large Kabir Jaiswal} \\

  \href{mailto:kabirjaiswal900@gmail.com}{EMAIL} |
  \href{https://www.linkedin.com/in/kabir-j}{LinkedIn} |
  \href{https://github.com/kabir12345}{Github} |
  +1 (980)-446-4489 |
  New York, NY (Open to Relocation)
\end{center}

\section{Summary}
Experienced Machine Learning Engineer with 4+ years in full-stack development, deploying production-grade ML solutions, and data pipelines.
Proficient in agile environments and cross-functional collaboration, adept at translating complex models into actionable insights.

\section{Technical Skills}
\resumeSubHeadingListStart
    \resumeItem{Deep Learning \& Efficient ML}{Quantization, Pruning, Neural Architecture Search (NAS), Knowledge Distillation, Efficient Model Architectures}
    \resumeItem{ML Frameworks}{TensorFlow, PyTorch, Hugging Face, ScikitLearn}
    \resumeItem{MLOps \& Deployment}{MLflow, ClearML, Apache Airflow, Docker, Kubernetes, Terraform, GitOps}
    \resumeItem{Programming Languages}{Python, C++, R}
    \resumeItem{Cloud Platforms}{AWS, GCP, Azure}
    \resumeItem{Software Engineering}{API Development, Version Control (Git), Embedded Systems, Digital Signal Processing}
\resumeSubHeadingListEnd

\section{Experience}
\vspace{1pt}
\resumeSubheading
  {Machine Learning Engineer}{January 2024 – May 2024}
  {BrandGuard AI}{New York, USA}
\resumeItemListStart
    \item Architected and implemented a resource-efficient custom vector embeddings for brands, optimized for 500 MFLOPs inference budget, leveraging quantization techniques and distillation methods.
    \item Enhanced BrandGPT RAG system using RagaTouille  with ColBERT-based dense retrieval, implementing context-aware filtering.
    \item Spearheaded end-to-end MLOps pipeline development, integrating GitOps principles, containerized deployments, and automated A/B testing frameworks, reducing deployment cycles by 30\%.
    \item Orchestrated comprehensive LLM fine-tuning and evaluation pipeline using Baserun for telemetry and RAGAS for multi-dimensional quality assessment, optimizing inference costs and mitigating hallucinations.
\resumeItemListEnd

\resumeSubheading
  {Machine Learning Engineer}{September 2023 – December 2023}
  {BrandGuard AI (Formerly Nova AI, Inc.) }{New York, USA}
\resumeItemListStart
    \item Launched synthetic data generation pipeline for ads using stable diffusion \& FastAPI. Optimized for parallel processing reducing asset creation time by 40\%.
    \item Conducted throughput analysis on platforms including MLflow and ClearML, utilizing AWS EC2 instances and ELB for load simulation.
    \item Orchestrated MLflow deployment on GCP using Infrastructure as Code (IaC) with Terraform, implementing a robust MLOps ecosystem. Integrated CI/CD pipelines, distributed monitoring and Git-based model versioning for reproducibility and governance.
    \item Spearheaded evaluation of enterprise-grade data labeling solutions, including Human Signal and V7 Labs, focusing on active learning strategies and quality control mechanisms.
\resumeItemListEnd

\resumeSubheading
  {Full Stack Developer}{January 2018 – March 2022}
  {Freelance}{New Delhi, India}
\resumeItemListStart
    \item Developed end-to-end products including ERPs, ticketing systems, and blockchain contracts using MERN stack.
    \item Implemented custom e-commerce solutions on the Liquid framework, integrating RESTful APIs, payment gateways.
    \item Orchestrated scalable deployments on AWS (EC2, S3, RDS) and GCP utilizing Docker for containerization and implementing CI/CD.
    \item Led a team of 5 developers using Agile methodologies, conducting code reviews, and implementing ELK stack for monitoring.
    \item Managed client relationships, including requirement gathering, regular progress updates, and facilitating knowledge transfer.
\resumeItemListEnd

\section{Projects}
\resumeSubHeadingListStart

    % \resumeProjectHeading
    %   {GoDesigner}{ — GitHub: \href{https://github.com/kabir12345/godesigner}{github.com/kabir12345/godesigner}}
    % \resumeItemListStart
    %       \item Engineered a Flask app for interior design recommendations using CLIP embeddings, integrating SAM and Google Shopping API, and deployed on AWS ECS for optimized proposal generation.
    % \resumeItemListEnd

    \resumeProjectHeading
      {Spatial Sense}{ — GitHub: \href{https://github.com/kabir12345/SpatialSenseWeb}{github.com/kabir12345/SpatialSenseWeb}}
    \resumeItemListStart
          \item Developed a real-time navigation aid using Qualcomm Gen 2 edge devices, with PEFT quantization of LLaVA-1.6 for reduced inference time and improved obstacle detection.
    \resumeItemListEnd

    \resumeProjectHeading
      {Qualitative Analysis of Quantization Techniques for Text Summarization}{ — GitHub: \href{https://github.com/kabir12345/LLM-PEFT-Optimization}{github.com/kabir12345/LLM-PEFT-Optimization}}
    \resumeItemListStart
          \item Optimized on-device LLM deployment by fine-tuning Mistral-7B and Llama on the CNN dataset, enhancing BLEU scores with minimal performance loss using LoRA and IA3.
    \resumeItemListEnd

\resumeSubHeadingListEnd

\section{Publications}
\resumeSubHeadingListStart

    \resumeProjectHeading
      {Fog Computing Concepts, Frameworks, and Applications}{ — Taylor and Francis Publications}
    \resumeItemListStart
          \item Co-Author and Academic Research Assistant (University of Petroleum and Energy Studies).
    \resumeItemListEnd
    
    \resumeProjectHeading
      {Apache Airflow Operator for KDB Integration}{ — GitHub: \href{https://pypi.org/project/airflow-kdb-provider/}{airflow-kdb-provider}}
    \resumeItemListStart
          \item Architected and developed an Apache Airflow operator for KDB using Python and the Astronomer open-source framework.
    \resumeItemListEnd

\resumeSubHeadingListEnd

\section{Education}
\resumeSubHeadingListStart
    \resumeSubheading
      {New York University}{May 2024}
      {Master of Science in Computer Engineering}{New York, USA}
    \resumeSubheading
      {University of Petroleum and Energy Studies}{July 2022}
      {Bachelor of Technology in Computer Science (Concentration: AI \& ML)}{Uttarakhand, India}
\resumeSubHeadingListEnd

\end{document}
```